%%%%%%%%%%%%%%% Permutacions %%%%%%%%%%%%%%%%%%%%%%%%%%%%%%%%%%%5
\section{Permutatcions}
\subsection*{Definicions}
\begin{itemize}
\item Grups de permutacions
	\begin{itemize}
	\item $X$ conjunt
	\item permutació és una bijecció de $X$ en $X$
	\end{itemize}
\item $S_X = \{$permutacions de $X\}$
	\begin{itemize}
	\item Composició d'aplicacions és grup
	\item $n \in \N^+, S_n$ grup de permutacions
	\item grup simètric de grau $n$
	\item $\sigma \in S_n$
		\subitem $\begin{pmatrix}1 & 2 & \dots & n\\ \sigma(1) & \sigma(2) & \dots & \sigma(n)\end{pmatrix}$ cardinal $n!$
	\end{itemize}
\item r-cicle
	\begin{itemize}
	\item $(k_1, k_2, \dots, k_r)$
	\end{itemize}
\item transposició
	\begin{itemize}
	\item 2-cicle
	\end{itemize}
\item 2 cicles disjunts
	\subitem $(k_1,,,), (h_1,,,)\quad \forall k_i \notin (h_1,,,)$
\item La paritat del nombre de factors en la descomposició d'una permutació en producte de transposicions queda determinat
	\subitem[Si] parell +1
	\subitem[Si] senar -1
\item $A_n$ subgrup $S_n$ que és +1
	\subitem Grup alternat de grau $n$
\end{itemize}

%%%%%%%%%%%%%%%%%%%%%%%% Demostracio %%%%%%%%%%%%%%%%%%%%%%%%%%%5
\subsection*{Demostracio}
like\\
Gran problema, com parlen i relacionen la composició d'aplicacions amb l'invers? No vindrié a ser el simètric?
\begin{itemize}
\item Tota permutació $S_n \neq Id$ és producte de cicles, disjunts 2 a 2
	\subitem Fem un cicle, el subdividim amb transposicions $(k_1, k_2), \dots, (k_{r-1}, k_r)$
	\subitem Busquem elements que no estiguin als trobats anteriorment, si és el cas tornem al pas anterior
\item Tota permutació és producte de transposicions
	\subitem N'hi ha prou amb veure que tot cicle és producte de transposicions.
\item Simètric en els r-ciclics
	\begin{itemize}
	\item $(k_1, k_2, \dots, k_r)^{-1} = (k_r, \dots, k_2, k_1)$
	\item invers d'una transposició és ella mateixa
	\end{itemize}
\item explicació
	\begin{itemize}
	\item $k_1$ l'envia a $k_2$, si girem, enviarà el $k_2$ al $k_1$ com voliem
	\item $(k_1, k_2)^{-1} = (k_2, k_1) = (k_1, k_2)$
	\end{itemize}
\item $(a, b, c) = (a, b)(b, c) = (a, b)(a, c)$
\item La identitat no és igual al producte d'un nombre senar de transposicions
	\begin{itemize}
	\item $Id$ no és producte de $2k +1$ transposicions $k \in \N$ per inducció
		\begin{enumerate}
		\item $k = 0$, està clar, $Id \neq$ transposició
		\item Suposem que no és producte per $2k -1$ i provem que tampoc ho és de $2k +1$
		\item $Id = t_1\dots t_{2k+1}$
		\item like, xunget pagina 10. concepte facil i llarga
		\end{enumerate}
	\end{itemize}
\item $t_{1-r} = \tau_{1-s}$ són transposicions
	\subitem $\Rightarrow r \equiv s \pmod{2}$
	\begin{itemize}
	\item $t_1\dots t_r = \tau_1\dots \tau_s \Rightarrow t_1\dots t_r \tau_s \dots \tau_1 = Id$
	\end{itemize}
\end{itemize}

\subsection*{Exemples}
\begin{itemize}
\item $S_3 = \left\{\begin{pmatrix}1&2&3\\1&2&3\end{pmatrix}, \begin{pmatrix}1&2&3\\2&1&3\end{pmatrix} \dots \right\}$\\
\item mante quiet 1 = $t_1$, el $t_2, t_3$, ho canvia tot 1 2 3 $s_1$ amb desordre 312 $s_2$
\item $\begin{pmatrix}1&2&3\\2&1&3\end{pmatrix}\begin{pmatrix}1&2&3\\3&2&1\end{pmatrix} = t_3 t_2 = \begin{pmatrix}1&2&3\\3&1&2\end{pmatrix} = s_2$
\item $\begin{pmatrix}1&2&3\\3&2&1\end{pmatrix}\begin{pmatrix}1&2&3\\2&1&3\end{pmatrix} = t_2 t_3 = \begin{pmatrix}1&2&3\\2&3&1\end{pmatrix} = s_1$
\item cicles
	\begin{itemize}
	\item $\begin{pmatrix}1&2&3\\2&3&1\end{pmatrix} = (1, 2, 3) = (2, 3, 1) = (1, 2)(2, 3)$
	\item $\begin{pmatrix}1&2&3&4&5\\2&3&1&5&4\end{pmatrix} = (1, 2, 3)(4, 5)$
	\end{itemize}
\end{itemize}
