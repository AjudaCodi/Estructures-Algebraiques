%%%%%%%%%%%%%%%%%%%%%%% Ordre %%%%%%%%%%%%%%%%%%%%%%5
\section{Ordre, relacions d'equivalència}
\subsection*{Definició}
\begin{itemize}
\item Ordre $|G|$
	\begin{itemize}
	\item $G$ grup finit
	\item És el nombre d'elements del conjunt
	\end{itemize}
\item Relació $D$ i $E$
	\begin{itemize}
	\item $G$ grup
	\item $H$ subgrup de $G$
		\begin{itemize}
		\item $xDy \Leftrightarrow x'y \in H$
		\item $yEx \Leftrightarrow yx' \in H$
		\end{itemize}
	\end{itemize}
\item Classes d'equivalència per $D$
	\begin{itemize}
	\item $[x] = \{y \in G | xDy\}$
	\item $xDy \Leftrightarrow x'y \in H \Leftrightarrow y = xh$ per a algún $h \in H$
	\item $xH = \{xh | h \in H\}$ ``$D$''
	\item $Hx = \{hx | h \in H\}$ ``$E$''
	\item $G/D$ conjunt quocient de $G$ per la relació $D$
	\item $G/E$ conjunt quocient de $G$ per la relació $E$
	\end{itemize}
\end{itemize}

\subsection*{Desmostracions}
\begin{itemize}
\item Les relacions $D$ i $E$ són d'equivalència
	\begin{itemize}
	\item $G$ grup
	\item $H$ subgrup de $G$
	\end{itemize}
	\begin{enumerate}
	\item $\forall x \in G\quad xDx \Rightarrow x'x = e \in H$ Reflexiva
	\item $x, y \in G\quad xDy \Rightarrow x'y \in H \Rightarrow y'x = (x'y)' \in H \Rightarrow yDx$ Simètrica
	\item $x, y, z \in G,
		\begin{rcases}
		xDy\\
		yDz
		\end{rcases}
		\Rightarrow
		\begin{rcases}
		x'y \in H\\
		y'z \in H
		\end{rcases}
		\Rightarrow
		x'z = (x'y)(y'z) \in H \Rightarrow xDz
		$ Transitiva
	\end{enumerate}
\item $G/E$ és bijectiu
	\begin{itemize}
	\item $
		\begin{array}{cc}
		H & \to xH \\
		h & \mapsto xh
		\end{array}
		$
	\item $
		\begin{array}{cc}
		H & \to Hx \\
		h & \mapsto hx
		\end{array}
		$
	\item $y \in xH \Leftrightarrow y = xh, h \in H \Leftrightarrow y' = hx' \Leftrightarrow y' \in Hx'$
		\subitem $y \mapsto y'$
	\item Tenim bijecció de $G/D$ i $G/E$
	\end{itemize}
\item Teorema de Lagrange
	\begin{itemize}
	\item $G$ grup finit
	\item $H$ subgrup de $G$
		\begin{itemize}
		\item $|G| = |H|\dot [G:H]$
		\end{itemize}
	\end{itemize}
\item Demostració like
	\begin{itemize}
	\item pagina 14 del pdf
%	\item Les classes d'equivalència per $D$ formen una partició de $G$, és a dir $G$ és reunió disjunta de les classes d'equivalència i a cada classe d'equivalència i a cada classe d'equivalè
	\end{itemize}
\end{itemize}

\subsection*{Exemples}
\begin{itemize}
\item Ordre
	\begin{itemize}
	\item $|S_n| = n!$
	\item $|\Z/m\Z| = m$
	\end{itemize}
\item Relació d'equivalència
	\begin{itemize}
	\item $G = S_3 = \{\Id, t_3, t_2, t_1, c_1, c_2\}$
	\item $H = \{\Id, t_1\}$
		\begin{itemize}
		\item Classes per $D$
			\begin{itemize}
			\item $[\Id] = H$
			\item $[t_2] = \{t_2, t_2t_1\} = \{t_2, c_2\}$
			\item $[t_3] = \{t_3, t_3t_1\} = \{t_3, c_1\}$
			\end{itemize}
		\item Classes per $E$
			\begin{itemize}
			\item $[\Id] = H$
			\item $[t_2] = \{t_2, t_1t_2\} = \{t_2, c_1\}$
			\item $[t_3] = \{t_3, t_1t_3\} = \{t_3, c_2\}$
			\end{itemize}
		\end{itemize}
	\end{itemize}
\item Índex de $G$ en $H$ el cardinal de $G/D$
	\begin{itemize}
	\item $[\Z:m\Z] = m$
	\item $[S_3: \{\Id, t_1\}] = 3$
	\end{itemize}
\end{itemize}
