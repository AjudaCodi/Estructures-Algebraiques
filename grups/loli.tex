\documentclass{article}
\usepackage[utf8]{inputenc}
\title{Estructures Algebraiques\\Grups}
\author{Sistach Reinoso, Arnau}

% nomes crec que serveix per la ela geminada (on sense el corrector dubto mai fer-ho anar 183 ·)
%\usepackage[catalan]{babel}

% perque quedi mes clar
\usepackage{color}



% faig anar equation*
\usepackage{amsmath}

\usepackage{amssymb}
\newcommand{\N}{\mathbb{N}}
\newcommand{\Z}{\mathbb{Z}}
\newcommand{\Q}{\mathbb{Q}}
\newcommand{\R}{\mathbb{R}}
\newcommand{\C}{\mathbb{C}}




% like kindle

\usepackage{microtype}     % microtypography, reduces hyphenation

\usepackage[papersize={3.6in,4.8in},hmargin=0.1in,vmargin={0.1in,0.1in}]{geometry}  % page geometry

\usepackage{fancyhdr}   % headers and footers
\pagestyle{fancy}
\fancyhead{}            % clear page header
\fancyfoot{}            % clear page footer

\setlength{\abovecaptionskip}{2pt} % space above captions 
\setlength{\belowcaptionskip}{0pt} % space below captions
\setlength{\textfloatsep}{2pt}     % space between last top float or first bottom float and the text
\setlength{\floatsep}{2pt}         % space left between floats
\setlength{\intextsep}{2pt}        % space left on top and bottom of an in-text float


% https://atlas.mat.ub.edu/personals/crespo/ApuntsEstructuresAlgebraiques.pdf
\begin{document}
\maketitle
Teoria de grups\\
grups finits hi ha teoremes forts\\
grups abelians\\
propietats de factorialitat\\
forçar-se pel primer parcial 45\% nota final = 2 parcial\\
dimarts mirar els nous exercisis\\
de la meta, fuertes xambo Introducción al algebra\\
\newpage

\section{Bàsic}
\subsection{Definicions}
\begin{itemize}
\item Grup
	\begin{itemize}
	\item Conjunt $G \neq \emptyset$
	\item Operació interna binària $\forall x, y, z \in G$
		\begin{enumerate}
		\item[assosiativa] $(xy)z = x(yz) $
		\item[el. neutre] $\exists e \in G: ex = xe = x$
		\item[simètric] $\forall x \in G, \exists x' \in G: xx' = x'x = e$
		\end{enumerate}
	\item grup $G$ commutatiu i/o abelià
		\begin{itemize}
		\item $\forall x, y \in G\quad xy = yx$
		\end{itemize}
	\end{itemize}
\item Subgrup
	\begin{itemize}
	\item $\emptyset \neq H \subset G$ grup
		\begin{enumerate}
		\item $\forall x, y \in H \Rightarrow xy \in H$
		\item $H$ és grup amb l'operació de $G$
		\end{enumerate}
	\end{itemize}
\end{itemize}

\subsection{Demostracions}
\subsection{Únic element d'un grup}
\begin{itemize}
\item Únic element d'un grup
	\begin{itemize}
	\item element neutre $e = ee' = e'$
	\item simètric $x' = x'e = x'(xx'') = (x'x)x'' = x''$
	\end{itemize}
\end{itemize}

\subsection{Propietats simples}
\begin{itemize}
\item Llei de simplificació - pels grups
	\begin{itemize}
	\item $\forall a, x, y \in G$ grup
		\begin{itemize}
		\item $ax = ay \Rightarrow x = y$
		\item $xa = ya \Rightarrow x = y$
		\end{itemize}
	\end{itemize}
\item Invers del producte
	\subitem $(xy)' = y'x'$
\end{itemize}
\subsection{Propietats iguals -subgrups}
\begin{enumerate}
\item $H$ és subgrup de $G$
\item $H$ compleix
	\begin{enumerate}
	\item $e \in H$
	\item $\forall x \in H \Rightarrow x' \in H$
	\item $\forall x, y \in H \Rightarrow xy \in H$
	\end{enumerate}
\item $\forall x, y \in H \Rightarrow xy' \in H$
\end{enumerate}
\subsubsection{demostració}
\begin{itemize}
\item $1. \to 2.$ és evident per definició de grup
\item $2. \to 3.$
	\begin{itemize}
	\item per (b) $y \in H \Rightarrow y' \in H$
	\item per (c) $x, y' \in H \Rightarrow xy' \in H$
	\end{itemize}
\item $3. \to 1.$
	\begin{itemize}
	\item $H \neq \emptyset$ ja que conté $x \in H$
	\item $\exists x \in H \Rightarrow$ $x, x \in H$ $\Rightarrow xx' = e \in H$
	\item $\forall y \in H \Rightarrow$ $e, y \in H$ $\Rightarrow ey' = y' \in H$
	\item $x, y' \in H \Rightarrow$ $x(y')' = xy \in H$
	\end{itemize}
\end{itemize}
\subsection{Propietats intersecció - like un ex.}
\begin{itemize}
\item $H_1, H_2$ subgrups de $G$ grup
	\subitem $H_1 \cap H_2$ és subgrup de $G$
\end{itemize}
\subsection{$H$ subgrup $(\Z, +)$, $H \equiv m\Z$ algún $m$}
\begin{itemize}
\item $H = \{0\} \Rightarrow H = 0 \Z$
\item $H \neq \{0\}$
	\begin{itemize}
	\item $m > 0$ més petit: $m \in H$
	\item $a \in H$
	\item $a = qm + r$ amb $0 \le r < m$
		\begin{itemize}
		\item $r = a - qm \in H \Rightarrow r = 0$
		\item $a = qm \in m\Z \Rightarrow H \subset m\Z$
		\end{itemize}
	\item $m\Z \subset H$ és obvi
%	\item $n \in H \Rightarrow -n \in H$
%		\subitem Conté elements estrictament positius
%		\subitem $m > 0$ més petit contingut $H \Rightarrow m\Z \subset H$
%	\item Sigui $a \in H$
%		\begin{itemize}
%		\item $a = qm + r, 0 \le r < m$
%		\item $r = a - qm \in H \Rightarrow r = 0$ i $a \in m\Z$
%		\end{itemize}
	\end{itemize}
\end{itemize}
\subsection{Proposició permutacions}
\begin{itemize}
\item Tota permutació $S_n \neq Id$ és producte de cicles, disjunts 2 a 2
	\subitem Demostració a la pàgina 9 -molt llarg i no ho entenc-
\item Tota permutació és producte de transposicions
	\subitem N'hi ha prou amb veure que tot cicle és producte de transposicions.
\end{itemize}

%%%%%%%%%%%%%%%%%%%%%%%%%%%%%%%%%%%%%%%%%%%%%%%%
\newpage\section{Exemples}
\subsection{grups abelians}
$(\Z, +), (\C, +), (\Q\setminus\{0\}, \times), (\Z/m\Z, +), ((\Z/m\Z)^*, \times)$
\subsection{grup no abelià}
$\forall K$ cos, $GL(n, K)$ i $n \ge 2$
\subsection{subgrup}
$m\Z$ és subgrup de $(\Z, +) \forall m \in \Z$
\subsection{permutacions}
\begin{itemize}
\item $S_3 = \left\{\begin{pmatrix}1&2&3\\1&2&3\end{pmatrix}, \begin{pmatrix}1&2&3\\2&1&3\end{pmatrix} \dots \right\}$\\
\item mante quiet 1 = $t_1$, el $t_2, t_3$, ho canvia tot 1 2 3 $s_1$ amb desordre 312 $s_2$
\item $\begin{pmatrix}1&2&3\\2&1&3\end{pmatrix}\begin{pmatrix}1&2&3\\3&2&1\end{pmatrix} = t_3 t_2 = \begin{pmatrix}1&2&3\\3&1&2\end{pmatrix} = s_2$
\item $\begin{pmatrix}1&2&3\\3&2&1\end{pmatrix}\begin{pmatrix}1&2&3\\2&1&3\end{pmatrix} = t_2 t_3 = \begin{pmatrix}1&2&3\\2&3&1\end{pmatrix} = s_1$
\end{itemize}
\subsection{cicles}
\begin{itemize}
\item $\begin{pmatrix}1&2&3\\2&3&1\end{pmatrix} = (1, 2, 3) = (2, 3, 1) = (1, 2)(2, 3)$
\item $\begin{pmatrix}1&2&3&4&5\\2&3&1&5&4\end{pmatrix} = (1, 2, 3)(4, 5) = (1, 2)(2, 3)(4, 5)$
\end{itemize}


%%%%%%%%%%%%%%% Permutacions %%%%%%%%%%%%%%%%%%%%%%%%%%%%%%%%%%%5
\section{Permutatcions}
\subsection{Grups de permutacions}
\begin{itemize}
\item $X$ conjunt
\item permutació és una bijecció de $X$ en $X$
\item $S_X = \{$permutacions de $X\}$
\item $n \in \N\setminus 0, S_n$ grup de permutacions de $\{1, 2,\dots , n\}$
\item grup simètric de grau $n$
\item $\begin{pmatrix}1 & 2 & \dots & n\\ \sigma(1) & \sigma(2) & \dots & \sigma(n)\end{pmatrix}$ cardinal $n!$
\end{itemize}
\subsection{r-cicle}
\begin{itemize}
\item $(k_1, k_2, \dots, k_r)$
\end{itemize}
\end{document}

dubtes
	- (2.4) Propietats intersecció
	- (2.5) H subgrup (Z, +), H = mZ algún m

inici del 11
morfismes de grup, encara no ho hem fet a clase
