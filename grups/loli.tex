\documentclass{article}
\usepackage[utf8]{inputenc}
\title{Estructures Algebraiques\\Grups}
\author{Sistach Reinoso, Arnau}

% nomes crec que serveix per la ela geminada (on sense el corrector dubto mai fer-ho anar 183 ·)
\usepackage[catalan]{babel}

% perque quedi mes clar
\usepackage{color}

% fer el rcases
\usepackage{mathtools}

% faig anar equation*
\usepackage{amsmath}


% declarant funcions
\DeclareMathOperator{\Ima}{Im}
\DeclareMathOperator{\SL}{SL}
\DeclareMathOperator{\Id}{Id}

\usepackage{amssymb}
\newcommand{\N}{\mathbb{N}}
\newcommand{\Z}{\mathbb{Z}}
\newcommand{\Q}{\mathbb{Q}}
\newcommand{\R}{\mathbb{R}}
\newcommand{\C}{\mathbb{C}}




% like kindle

\usepackage{microtype}     % microtypography, reduces hyphenation

\usepackage[papersize={3.6in,4.8in},hmargin=0.1in,vmargin={0.1in,0.1in}]{geometry}  % page geometry

\usepackage{fancyhdr}   % headers and footers
\pagestyle{fancy}
\fancyhead{}            % clear page header
\fancyfoot{}            % clear page footer

\setlength{\abovecaptionskip}{2pt} % space above captions 
\setlength{\belowcaptionskip}{0pt} % space below captions
\setlength{\textfloatsep}{2pt}     % space between last top float or first bottom float and the text
\setlength{\floatsep}{2pt}         % space left between floats
\setlength{\intextsep}{2pt}        % space left on top and bottom of an in-text float


% https://atlas.mat.ub.edu/personals/crespo/ApuntsEstructuresAlgebraiques.pdf
\begin{document}
\maketitle
Teoria de grups\\
grups finits hi ha teoremes forts\\
grups abelians\\
propietats de factorialitat\\
forçar-se pel primer parcial 45\% nota final = 2 parcial\\
dimarts mirar els nous exercisis\\
de la meta, fuertes xambo Introducción al algebra\\
$0 \in \Z$,
$0 \notin {\mathbb Z}^+$
\newpage
\tableofcontents
\newpage

\section{Bàsic}
\subsection*{Definicions}
\begin{itemize}
\item Grup
	\begin{itemize}
	\item Conjunt $G \neq \emptyset$
	\item Operació interna binària $\forall x, y, z \in G$
		\begin{enumerate}
		\item[assosiativa] $(xy)z = x(yz) $
		\item[el. neutre] $\exists e \in G: ex = xe = x$
		\item[simètric] $\forall x \in G, \exists x' \in G: xx' = x'x = e$
		\end{enumerate}
	\item grup $G$ commutatiu i/o abelià
		\begin{itemize}
		\item $\forall x, y \in G\quad xy = yx$
		\end{itemize}
	\end{itemize}
\item Subgrup
	\begin{itemize}
	\item $\emptyset \neq H \subset G$ grup
		\begin{enumerate}
		\item $\forall x, y \in H \Rightarrow xy \in H$
		\item $H$ és grup amb l'operació de $G$
		\end{enumerate}
	\end{itemize}
\end{itemize}

\subsection*{Demostracions}
\begin{itemize}
\item Únic element d'un grup
	\begin{itemize}
	\item element neutre $e = ee' = e'$
	\item simètric $x' = x'e = x'(xx'') = (x'x)x'' = x''$
	\end{itemize}
\item Llei de simplificació - pels grups
	\begin{itemize}
	\item $\forall a, x, y \in G$ grup
		\begin{itemize}
		\item $ax = ay \Rightarrow x = y$
		\item $xa = ya \Rightarrow x = y$
		\end{itemize}
	\end{itemize}
\item Invers del producte
	\subitem $(xy)' = y'x'$
\item Propietats iguals -subgrups
	\begin{enumerate}
	\item $H$ és subgrup de $G$
	\item $H$ compleix
		\begin{enumerate}
		\item $e \in H$
		\item $\forall x \in H \Rightarrow x' \in H$
		\item $\forall x, y \in H \Rightarrow xy \in H$
		\end{enumerate}
	\item $\forall x, y \in H \Rightarrow xy' \in H$
	\end{enumerate}
\item demostració
	\begin{itemize}
	\item $1. \to 2.$ és evident per definició de grup
	\item $2. \to 3.$
		\begin{itemize}
		\item $\forall y \in H \Rightarrow y' \in H$
		\item $\forall x \in H, y' \in H \Rightarrow xy' \in H$
		\end{itemize}
	\item $3. \to 1.$
		\begin{itemize}
		\item $H \neq \emptyset$ ja que $x \in H$
		\item $\exists x \in H \Rightarrow$ $x, x \in H$ $\Rightarrow xx' = e \in H$
		\item $\forall y \in H \Rightarrow$ $e, y \in H$ $\Rightarrow ey' = y' \in H$
		\item $x, y' \in H \Rightarrow$ $x(y')' = xy \in H$
		\end{itemize}
	\end{itemize}
\item Propietats intersecció
	\begin{itemize}
	\item $H_1, H_2$ subgrups de $G$ grup
		\subitem $H_1 \cap H_2$ és subgrup de $G$
	\end{itemize}
\item Demostració
	\begin{itemize}
	\item $e_1 \in H_1, e_2 \in H_2, e \in G$
		\subitem $e_1 = e_2 = e$
	\item $\forall x \in H_1 \cap H_2 \Rightarrow x' \in H_1 \cap H_2$
	\item $\forall x, y \in H_1 \cap H_2 \overset{?}{\Rightarrow} xy \in H_1 \cap H_2$
		\begin{enumerate}
		\item $\forall x \in H_1, H_2$
		\item $\forall y \in H_1$
			\begin{itemize}
			\item $y \in H_2$
				\subitem $x, y \in H_1 \cap H_2 \Rightarrow xy \in H_1 \cap H_2$
			\item $y \notin H_2$
				\subitem $y \notin H_1 \cap H_2$
			\end{itemize}
		\end{enumerate}
	\end{itemize}
\item $H$ subgrup $(\Z, +)$, $H \equiv m\Z$ algún $m$
	\begin{itemize}
	\item $H = \{0\} \Rightarrow H = 0 \Z$
	\item $H \neq \{0\}$
		\begin{itemize}
		\item $m > 0$ més petit: $m \in H$
		\item $a \in H$
		\item $a = qm + r$ amb $0 \le r < m$
			\begin{itemize}
			\item $r = a - qm \in H \Rightarrow r = 0$
			\item $a = qm \in m\Z \Rightarrow H \subset m\Z$
			\end{itemize}
		\item $n \in H \Rightarrow -n \in H$
			\begin{itemize}
			\item Conté elements positius
			\item $0 < m \in H$ més petit $\Rightarrow m\Z \subset H$
			\end{itemize}
		\end{itemize}
	\end{itemize}
\end{itemize}

%%%%%%%%%%%%%%%%%%%%%%%%%%%%%%%%%%%%%%%%%%%%%%%%
\subsection*{Exemples}
\begin{itemize}
\item Grups abelians
	\subitem $(\Z, +), (\C, +), (\Q\setminus\{0\}, \times), (\Z/m\Z, +), ((\Z/m\Z)^*, \times)$
\item Grup no abelià
	\subitem $\forall K$ cos, $GL(n, K)$ i $n \ge 2$
\item Subgrup
	\subitem $m\Z$ és subgrup de $(\Z, +) \forall m \in \Z$
\end{itemize}


%%%%%%%%%%%%%%% Permutacions %%%%%%%%%%%%%%%%%%%%%%%%%%%%%%%%%%%5
\section{Permutatcions}
\subsection*{Definicions}
\begin{itemize}
\item Grups de permutacions
	\begin{itemize}
	\item $X$ conjunt
	\item permutació és una bijecció de $X$ en $X$
	\end{itemize}
\item $S_X = \{$permutacions de $X\}$
	\begin{itemize}
	\item Composició d'aplicacions és grup
	\item $n \in \N^+, S_n$ grup de permutacions
	\item grup simètric de grau $n$
	\item $\sigma \in S_n$
		\subitem $\begin{pmatrix}1 & 2 & \dots & n\\ \sigma(1) & \sigma(2) & \dots & \sigma(n)\end{pmatrix}$ cardinal $n!$
	\end{itemize}
\item r-cicle
	\begin{itemize}
	\item $(k_1, k_2, \dots, k_r)$
	\end{itemize}
\item transposició
	\begin{itemize}
	\item 2-cicle
	\end{itemize}
\item 2 cicles disjunts
	\subitem $(k_1,,,), (h_1,,,)\quad \forall k_i \notin (h_1,,,)$
\item La paritat del nombre de factors en la descomposició d'una permutació en producte de transposicions queda determinat
	\subitem[Si] parell +1
	\subitem[Si] senar -1
\item $A_n$ subgrup $S_n$ que és +1
	\subitem Grup alternat de grau $n$
\end{itemize}

%%%%%%%%%%%%%%%%%%%%%%%% Demostracio %%%%%%%%%%%%%%%%%%%%%%%%%%%5
\subsection*{Demostracio}
like\\
Gran problema, com parlen i relacionen la composició d'aplicacions amb l'invers? No vindrié a ser el simètric?
\begin{itemize}
\item Tota permutació $S_n \neq Id$ és producte de cicles, disjunts 2 a 2
	\subitem Fem un cicle, el subdividim amb transposicions $(k_1, k_2), \dots, (k_{r-1}, k_r)$
	\subitem Busquem elements que no estiguin als trobats anteriorment, si és el cas tornem al pas anterior
\item Tota permutació és producte de transposicions
	\subitem N'hi ha prou amb veure que tot cicle és producte de transposicions.
\item Simètric en els r-ciclics
	\begin{itemize}
	\item $(k_1, k_2, \dots, k_r)^{-1} = (k_r, \dots, k_2, k_1)$
	\item invers d'una transposició és ella mateixa
	\end{itemize}
\item explicació
	\begin{itemize}
	\item $k_1$ l'envia a $k_2$, si girem, enviarà el $k_2$ al $k_1$ com voliem
	\item $(k_1, k_2)^{-1} = (k_2, k_1) = (k_1, k_2)$
	\end{itemize}
\item $(a, b, c) = (a, b)(b, c) = (a, b)(a, c)$
\item La identitat no és igual al producte d'un nombre senar de transposicions
	\begin{itemize}
	\item $Id$ no és producte de $2k +1$ transposicions $k \in \N$ per inducció
		\begin{enumerate}
		\item $k = 0$, està clar, $Id \neq$ transposició
		\item Suposem que no és producte per $2k -1$ i provem que tampoc ho és de $2k +1$
		\item $Id = t_1\dots t_{2k+1}$
		\item like, xunget pagina 10. concepte facil i llarga
		\end{enumerate}
	\end{itemize}
\item $t_{1-r} = \tau_{1-s}$ són transposicions
	\subitem $\Rightarrow r \equiv s \pmod{2}$
	\begin{itemize}
	\item $t_1\dots t_r = \tau_1\dots \tau_s \Rightarrow t_1\dots t_r \tau_s \dots \tau_1 = Id$
	\end{itemize}
\end{itemize}

\subsection*{Exemples}
\begin{itemize}
\item $S_3 = \left\{\begin{pmatrix}1&2&3\\1&2&3\end{pmatrix}, \begin{pmatrix}1&2&3\\2&1&3\end{pmatrix} \dots \right\}$\\
\item mante quiet 1 = $t_1$, el $t_2, t_3$, ho canvia tot 1 2 3 $s_1$ amb desordre 312 $s_2$
\item $\begin{pmatrix}1&2&3\\2&1&3\end{pmatrix}\begin{pmatrix}1&2&3\\3&2&1\end{pmatrix} = t_3 t_2 = \begin{pmatrix}1&2&3\\3&1&2\end{pmatrix} = s_2$
\item $\begin{pmatrix}1&2&3\\3&2&1\end{pmatrix}\begin{pmatrix}1&2&3\\2&1&3\end{pmatrix} = t_2 t_3 = \begin{pmatrix}1&2&3\\2&3&1\end{pmatrix} = s_1$
\item cicles
	\begin{itemize}
	\item $\begin{pmatrix}1&2&3\\2&3&1\end{pmatrix} = (1, 2, 3) = (2, 3, 1) = (1, 2)(2, 3)$
	\item $\begin{pmatrix}1&2&3&4&5\\2&3&1&5&4\end{pmatrix} = (1, 2, 3)(4, 5)$
	\end{itemize}
\end{itemize}

%%%%%%%%%%%%%%%%%%%%%% Morfisme de grups %%%%%%%%%%%%%%%%%%%%%%%%%%%%%%%%%%%5
\section{Morfismes de grups}
\subsection*{Definicions}
\begin{itemize}
\item Morfisme de grups $f$
	\begin{itemize}
	\item $G, G'$ grups
	\item $f:G\to G'$ aplicació
		\subitem $\forall x, y \in G\quad f(xy) = f(x)f(y)$
	\end{itemize}
\item nucli
	\begin{itemize}
	\item $\ker{f} = \{x \in G | f(x) = e'\}$ subgrup $G$
	\end{itemize}
\item imatge
	\begin{itemize}
	\item $\Ima{f} = \{f(x)| x \in G\}$ subgrup $G'$
	\end{itemize}
\item Monomorfisme de grups -injectiu
\item Epimorfisme de gurps -exhaustiu
\item Isomorfisme de grups -bijectiu $\simeq$
\item Endomorfisme de grups -$G \to G$
\item Automorfisme de $G$ -endomorfisme bijectiu

\item $\ker{(\det)} = \SL(n, K)$ grup especial lineal
\item $\ker{(\varepsilon)} = A_n$ grup alternat
\item $\ker{(\pi)} = m\Z$ $\quad \pi:\Z \to \Z/m\Z$
\end{itemize}

\subsection*{Demostració}
\begin{itemize}
\item Proposició
	\begin{itemize}
	\item $G, G'$ són grups
	\item $e \in G$ $e' \in G'$ elements neutres
		\begin{enumerate}
		\item $f(e) = e'$
		\item $f(x^{-1}) = f(x)^{-1}$
		\end{enumerate}
	\end{itemize}
\item demostració
	\begin{enumerate}
	\item $f(e) = f(ee) = f(e)f(e) \Rightarrow f(e) = e'$
	\item $f(x^{-1})f(x) = f(x^{-1}x) = f(e) = e'$
	\end{enumerate}
\item Prop
	\begin{itemize}
	\item $G, G', G''$ són grups
	\item $f: G \to G'$ morfisme de grups
	\item $g: G' \to G''$ morfisme de grups
		\begin{itemize}
		\item $\Rightarrow g \circ f: G \to G''$ és morfisme de grups
		\end{itemize}
	\end{itemize}
\item $f$ és isomorfisme, $f^{-1}$ també
	\begin{itemize}
	\item $x, y \in G'$
	\item $f(f'(xy)) = f(f'(x)f'(x)) = ff'(x)ff'(y) = xy$
	\end{itemize}
\item $f$ injectiu $\Leftrightarrow \ker{f} = \{e\}$
	\begin{itemize}
	\item[$\Rightarrow$] $x\! \in G, x \in \ker(f)\! \Rightarrow f(x) = e' =\! f(e) \Rightarrow x = e$
	\item[$\Leftarrow$] $x, y \in G, f(x) = f(y) \Rightarrow f(x)f(y)' = e' \Rightarrow xy' \in \ker{f} = \{e\} \Rightarrow x = y$
	\end{itemize}
\end{itemize}

\subsection*{Exemples}
\begin{itemize}
\item Morfismes de grups
	\begin{itemize}
	\item $det: GL(n, \R) \to \R^*$
	\item $\varepsilon: S_n \to \{\pm 1\}$
	\item $\pi: (\Z, +) \to (\Z/m\Z, +), a \mapsto [a]$
	\end{itemize}
\item Automorfisme
	\begin{itemize}
	\item $f_x:G\to G'$
	\item $y \mapsto xyx'$
		\begin{itemize}
		\item $f_x(yz) = xyzx' = (xyx')(xzx')$
		\item $(xyx')(xzx') = f_x(y)f_x(z)$
		\end{itemize}
	\end{itemize}
\end{itemize}

%%%%%%%%%%%%%%%%%%%%%%% Ordre %%%%%%%%%%%%%%%%%%%%%%5
\section{Ordre, relacions d'equivalència}
\subsection*{Definició}
\begin{itemize}
\item Ordre $|G|$
	\begin{itemize}
	\item $G$ grup finit
	\item És el nombre d'elements del conjunt
	\end{itemize}
\item Relació $D$ i $E$
	\begin{itemize}
	\item $G$ grup
	\item $H$ subgrup de $G$
		\begin{itemize}
		\item $xDy \Leftrightarrow x'y \in H$
		\item $yEx \Leftrightarrow yx' \in H$
		\end{itemize}
	\end{itemize}
\item Classes d'equivalència per $D$
	\begin{itemize}
	\item $[x] = \{y \in G | xDy\}$
	\item $xDy \Leftrightarrow x'y \in H \Leftrightarrow y = xh$ per a algún $h \in H$
	\item $xH = \{xh | h \in H\}$ ``$D$''
	\item $Hx = \{hx | h \in H\}$ ``$E$''
	\item $G/D$ conjunt quocient de $G$ per la relació $D$
	\item $G/E$ conjunt quocient de $G$ per la relació $E$
	\end{itemize}
\end{itemize}

\subsection*{Desmostracions}
\begin{itemize}
\item Les relacions $D$ i $E$ són d'equivalència
	\begin{itemize}
	\item $G$ grup
	\item $H$ subgrup de $G$
	\end{itemize}
	\begin{enumerate}
	\item $\forall x \in G\quad xDx \Rightarrow x'x = e \in H$ Reflexiva
	\item $x, y \in G\quad xDy \Rightarrow x'y \in H \Rightarrow y'x = (x'y)' \in H \Rightarrow yDx$ Simètrica
	\item $x, y, z \in G,
		\begin{rcases}
		xDy\\
		yDz
		\end{rcases}
		\Rightarrow
		\begin{rcases}
		x'y \in H\\
		y'z \in H
		\end{rcases}
		\Rightarrow
		x'z = (x'y)(y'z) \in H \Rightarrow xDz
		$ Transitiva
	\end{enumerate}
\item $G/E$ és bijectiu
	\begin{itemize}
	\item $
		\begin{array}{cc}
		H & \to xH \\
		h & \mapsto xh
		\end{array}
		$
	\item $
		\begin{array}{cc}
		H & \to Hx \\
		h & \mapsto hx
		\end{array}
		$
	\item $y \in xH \Leftrightarrow y = xh, h \in H \Leftrightarrow y' = hx' \Leftrightarrow y' \in Hx'$
		\subitem $y \mapsto y'$
	\item Tenim bijecció de $G/D$ i $G/E$
	\end{itemize}
\item Teorema de Lagrange
	\begin{itemize}
	\item $G$ grup finit
	\item $H$ subgrup de $G$
		\begin{itemize}
		\item $|G| = |H|\dot [G:H]$
		\end{itemize}
	\end{itemize}
\item Demostració like
	\begin{itemize}
	\item pagina 14 del pdf
%	\item Les classes d'equivalència per $D$ formen una partició de $G$, és a dir $G$ és reunió disjunta de les classes d'equivalència i a cada classe d'equivalència i a cada classe d'equivalè
	\end{itemize}
\end{itemize}

\subsection*{Exemples}
\begin{itemize}
\item Ordre
	\begin{itemize}
	\item $|S_n| = n!$
	\item $|\Z/m\Z| = m$
	\end{itemize}
\item Relació d'equivalència
	\begin{itemize}
	\item $G = S_3 = \{\Id, t_3, t_2, t_1, c_1, c_2\}$
	\item $H = \{\Id, t_1\}$
		\begin{itemize}
		\item Classes per $D$
			\begin{itemize}
			\item $[\Id] = H$
			\item $[t_2] = \{t_2, t_2t_1\} = \{t_2, c_2\}$
			\item $[t_3] = \{t_3, t_3t_1\} = \{t_3, c_1\}$
			\end{itemize}
		\item Classes per $E$
			\begin{itemize}
			\item $[\Id] = H$
			\item $[t_2] = \{t_2, t_1t_2\} = \{t_2, c_1\}$
			\item $[t_3] = \{t_3, t_1t_3\} = \{t_3, c_2\}$
			\end{itemize}
		\end{itemize}
	\end{itemize}
\item Índex de $G$ en $H$ el cardinal de $G/D$
	\begin{itemize}
	\item $[\Z:m\Z] = m$
	\item $[S_3: \{\Id, t_1\}] = 3$
	\end{itemize}
\end{itemize}

\section{Subgrups normals. Grup quocient}
\subsection*{Definicions}
\begin{itemize}
\item $T$ compatible amb l'operació de $G$
	\begin{itemize}
	\item $G$ grup
	\item $T$ relació d'equivalència de $G$
		\begin{itemize}
		\item $x, y, x', y' \in G$
		\item 	$
			\begin{rcases}
			xTx'\\yTy'
			\end{rcases}
			\Rightarrow xyTx'y'
			$
		\end{itemize}
	\end{itemize}
\end{itemize}

\subsection*{Demostracions}
\begin{itemize}
\item Proposició
	\begin{itemize}
	\item[Si] $T$ és relació d'equivalència compatible amb l'operació de $G$
	\item[$\Rightarrow$] $G/T$ és grup amb l'operació definida per
		\subitem $[x][y] = [xy]$
	\end{itemize}
\item Demostració
	\begin{itemize}
	\item L'operació esta ben definida
		\begin{itemize}
		\item Associativa
		\item Element neutre $[e]$
		\item Simètric $[x']$
		\end{itemize}
	\end{itemize}
\item Proposició
	\begin{itemize}
	\item $G$ grup
	\item $H$ subgrup de $G$
	\item $D, E$ relacions definides a partir $H$
	\item $\forall x \in G$
		\begin{enumerate}
		\item $xH = Hx$
		\item $xHx' = H$
		\item $xHx' \subset H$
		\item $D$ és compatible amb l'operació de $G$
		\item $E$ és compatible amb l'operació de $G$
		\end{enumerate}
	\end{itemize}
\item Demostració
	\begin{itemize}
	\item $1 \to 2$
		\begin{itemize}
		\item $h, h_1 \in H$
		\item $xhx' = (h_1x)x' = h_1 (xx') = h_1 \in H$
		\end{itemize}
	\item $2 \to 3$ Correcte
	\item $3 \to 1$
		\begin{itemize}
		\item $xHx' \subset H \Rightarrow xH \subset Hx$
		\item $x'Hx \subset H \Rightarrow Hx \subset xH$
		\end{itemize}
	\end{itemize}
\end{itemize}

\subsection*{Exemples}
\end{document}
%\section{Subgrups normals. Grup quocient}
%\subsection*{Definicions}
%\subsection*{Demostracions}
%\subsection*{Exemples}

dubtes
	- (2.5) H subgrup (Z, +), H = mZ algún m

inici del 11
morfismes de grup, encara no ho hem fet a clase
